%% Generated by Sphinx.
\def\sphinxdocclass{report}
\documentclass[letterpaper,10pt,english]{sphinxmanual}
\ifdefined\pdfpxdimen
   \let\sphinxpxdimen\pdfpxdimen\else\newdimen\sphinxpxdimen
\fi \sphinxpxdimen=.75bp\relax

\PassOptionsToPackage{warn}{textcomp}
\usepackage[utf8]{inputenc}
\ifdefined\DeclareUnicodeCharacter
% support both utf8 and utf8x syntaxes
  \ifdefined\DeclareUnicodeCharacterAsOptional
    \def\sphinxDUC#1{\DeclareUnicodeCharacter{"#1}}
  \else
    \let\sphinxDUC\DeclareUnicodeCharacter
  \fi
  \sphinxDUC{00A0}{\nobreakspace}
  \sphinxDUC{2500}{\sphinxunichar{2500}}
  \sphinxDUC{2502}{\sphinxunichar{2502}}
  \sphinxDUC{2514}{\sphinxunichar{2514}}
  \sphinxDUC{251C}{\sphinxunichar{251C}}
  \sphinxDUC{2572}{\textbackslash}
\fi
\usepackage{cmap}
\usepackage[T1]{fontenc}
\usepackage{amsmath,amssymb,amstext}
\usepackage{babel}



\usepackage{times}
\expandafter\ifx\csname T@LGR\endcsname\relax
\else
% LGR was declared as font encoding
  \substitutefont{LGR}{\rmdefault}{cmr}
  \substitutefont{LGR}{\sfdefault}{cmss}
  \substitutefont{LGR}{\ttdefault}{cmtt}
\fi
\expandafter\ifx\csname T@X2\endcsname\relax
  \expandafter\ifx\csname T@T2A\endcsname\relax
  \else
  % T2A was declared as font encoding
    \substitutefont{T2A}{\rmdefault}{cmr}
    \substitutefont{T2A}{\sfdefault}{cmss}
    \substitutefont{T2A}{\ttdefault}{cmtt}
  \fi
\else
% X2 was declared as font encoding
  \substitutefont{X2}{\rmdefault}{cmr}
  \substitutefont{X2}{\sfdefault}{cmss}
  \substitutefont{X2}{\ttdefault}{cmtt}
\fi


\usepackage[Bjarne]{fncychap}
\usepackage{sphinx}

\fvset{fontsize=\small}
\usepackage{geometry}


% Include hyperref last.
\usepackage{hyperref}
% Fix anchor placement for figures with captions.
\usepackage{hypcap}% it must be loaded after hyperref.
% Set up styles of URL: it should be placed after hyperref.
\urlstyle{same}

\addto\captionsenglish{\renewcommand{\contentsname}{Table of contents}}

\usepackage{sphinxmessages}
\setcounter{tocdepth}{1}



\title{Kamstrup OmniPower wm\sphinxhyphen{}bus metering}
\date{Oct 19, 2020}
\release{development}
\author{Team 3, E5PRO5 2020, Aarhus Uni., School of Engineering}
\newcommand{\sphinxlogo}{\vbox{}}
\renewcommand{\releasename}{Release}
\makeindex
\begin{document}

\pagestyle{empty}
\sphinxmaketitle
\pagestyle{plain}
\sphinxtableofcontents
\pagestyle{normal}
\phantomsection\label{\detokenize{index::doc}}



\chapter{OmniPower implementation}
\label{\detokenize{omnipower:module-OmniPower.OmniPower}}\label{\detokenize{omnipower:omnipower-implementation}}\label{\detokenize{omnipower::doc}}\index{module@\spxentry{module}!OmniPower.OmniPower@\spxentry{OmniPower.OmniPower}}\index{OmniPower.OmniPower@\spxentry{OmniPower.OmniPower}!module@\spxentry{module}}

\section{Parse Kamstrup OmniPower wm\sphinxhyphen{}bus telegrams}
\label{\detokenize{omnipower:parse-kamstrup-omnipower-wm-bus-telegrams}}\begin{quote}\begin{description}
\item[{platform}] \leavevmode
Python 3.5.10 on Linux, OS X

\item[{synopsis}] \leavevmode
Implements parsing functionality for C1 telegrams and log handling for data series

\item[{author}] \leavevmode
Janus Bo Andersen

\item[{date}] \leavevmode
14 October 2020

\end{description}\end{quote}


\subsection{Overview}
\label{\detokenize{omnipower:overview}}\begin{itemize}
\item {} 
This module implements parsing for the Kamstrup OmniPower meter, single\sphinxhyphen{}phase.

\item {} 
The meter sends wm\sphinxhyphen{}bus C1 (compact one\sphinxhyphen{}way) telegrams.

\item {} 
Telegrams on wm\sphinxhyphen{}bus are little\sphinxhyphen{}endian, i.e. LSB first.

\item {} 
The meter sends 1 long and 7 short telegrams, and then repeats.

\item {} 
Long telegrams include data record headers (DRH) and data, that is DIF/VIF codes + data.

\item {} 
Short telegrams only include data.

\end{itemize}


\subsection{Telegram fields}
\label{\detokenize{omnipower:telegram-fields}}
In a telegram C1 telegram, the data fields are:


\begin{savenotes}\sphinxattablestart
\centering
\begin{tabulary}{\linewidth}[t]{|T|T|T|T|T|T|}
\hline
\sphinxstyletheadfamily 
\#
&\sphinxstyletheadfamily 
Byte\#
&\sphinxstyletheadfamily 
Bytes
&\sphinxstyletheadfamily 
M\sphinxhyphen{}bus field
&\sphinxstyletheadfamily 
Description
&\sphinxstyletheadfamily 
Expected value (little\sphinxhyphen{}endian)
\\
\hline
0
&
0
&
1
&
L
&
Telegram length
&
0x27 (39 bytes, short frame), or
0x2D (45 bytes, long frame)
\\
\hline
1
&
1
&
1
&
C
&
Control field (type and purpose of message)
&
0x44 (SND\_NR)
\\
\hline
2
&
2\sphinxhyphen{}3
&
2
&
M
&
Manufacturer ID (official ID code)
&
0x2D2C (KAM)
\\
\hline
3
&
4\sphinxhyphen{}7
&
4
&
A
&
Address (meter serial number)
&
0x57686632 (big\sphinxhyphen{}endian:32666857)
\\
\hline
4
&
8
&
1
&
Ver.
&
Version number of the wm\sphinxhyphen{}bus firmware
&
0x30
\\
\hline
5
&
9
&
1
&
Medium
&
Type / medium of meter
&
0x02 (Electricity)
\\
\hline
6
&
10
&
1
&
CI
&
Control Information
&
0x8D (Extended Link Layer 2)
\\
\hline
7
&
11
&
1
&
CC
&
Communication Control
&
0x20 (Slow response sync.)
\\
\hline
8
&
12
&
1
&
ACC
&
Access field
&
Varies
\\
\hline
9
&
13\sphinxhyphen{}16
&
4
&
AES CTR
&
AES counter
&
Varies, used for decryption
\\
\hline
10
&
17\sphinxhyphen{}39
17\sphinxhyphen{}45
&
23
29
&
Data
&
Contains AES\sphinxhyphen{}encrypted data frame,
varying for short and long frames
&
Encrypted data
\\
\hline
11
&&
2
&
CRC16
&
CRC16 check
&\\
\hline
\end{tabulary}
\par
\sphinxattableend\end{savenotes}

The fields 0\sphinxhyphen{}9 of the telegram can be unpacked using the little\sphinxhyphen{}endian format \sphinxtitleref{\textless{}BBHIBBBBB}, where
\begin{itemize}
\item {} 
\sphinxtitleref{\textless{}} marks little\sphinxhyphen{}endian,

\item {} 
\sphinxtitleref{B} is an unsigned 1 byte (char),

\item {} 
\sphinxtitleref{H} is an unsigned 2 byte (short),

\item {} 
\sphinxtitleref{I} is an unsigned 4 byte (int)

\end{itemize}


\subsection{Telegram examples}
\label{\detokenize{omnipower:telegram-examples}}
Encrypted short telegrams:


\begin{savenotes}\sphinxattablestart
\centering
\begin{tabulary}{\linewidth}[t]{|T|T|T|T|T|T|T|T|T|T|T|T|}
\hline
\sphinxstyletheadfamily 
L
&\sphinxstyletheadfamily 
C
&\sphinxstyletheadfamily 
M
&\sphinxstyletheadfamily 
A
&\sphinxstyletheadfamily 
Ver
&\sphinxstyletheadfamily 
Med
&\sphinxstyletheadfamily 
CI
&\sphinxstyletheadfamily 
CC
&\sphinxstyletheadfamily 
ACC
&\sphinxstyletheadfamily 
AES CTR
&\sphinxstyletheadfamily 
Encrypted payload
&\sphinxstyletheadfamily 
CRC 16
\\
\hline
27
&
44
&
2D 2C
&
5768 6632
&
30
&
02
&
8D
&
20
&
2E
&
2187 0320
&
D3A4F149 B1B8F578 3DF7434B 8A66A557 86499ABE 7BAB59
&
xxxx
\\
\hline
27
&
44
&
2d 2c
&
5768 6632
&
30
&
02
&
8d
&
20
&
63
&
60dd 0320
&
c42b87f4 6fc048d4 2498b44b 5e34f083 e93e6af1 617631
&
3d9c
\\
\hline
27
&
44
&
2d 2c
&
5768 6632
&
30
&
02
&
8d
&
20
&
8e
&
11de 0320
&
188851bd c4b72dd3 c2954a34 1be369e9 089b4eb3 858169
&
494e
\\
\hline
\end{tabulary}
\par
\sphinxattableend\end{savenotes}

Encrypted long telegrams:


\begin{savenotes}\sphinxattablestart
\centering
\begin{tabulary}{\linewidth}[t]{|T|T|T|T|T|T|T|T|T|T|T|T|}
\hline
\sphinxstyletheadfamily 
L
&\sphinxstyletheadfamily 
C
&\sphinxstyletheadfamily 
M
&\sphinxstyletheadfamily 
A
&\sphinxstyletheadfamily 
Ver
&\sphinxstyletheadfamily 
Med
&\sphinxstyletheadfamily 
CI
&\sphinxstyletheadfamily 
CC
&\sphinxstyletheadfamily 
ACC
&\sphinxstyletheadfamily 
AES CTR
&\sphinxstyletheadfamily 
Encrypted payload
&\sphinxstyletheadfamily 
CRC 16
\\
\hline
2D
&
44
&
2D 2C
&
5768 6632
&
30
&
02
&
8D
&
20
&
64
&
61DD 0320
&
38931d14 b405536e 0250592f 8b908138 d58602ec a676ff79 e0caf0b1 4d
&
0e7d
\\
\hline
\end{tabulary}
\par
\sphinxattableend\end{savenotes}


\subsection{Decryption}
\label{\detokenize{omnipower:decryption}}
The AES\sphinxhyphen{}128 Mode CTR (or CBC\sphinxhyphen{}IV) decryption prefix is built from some of the fields (m\sphinxhyphen{}bus mode 8). See EN 13757\sphinxhyphen{}7:2018.
It can be packed using the format \sphinxtitleref{\textless{}HIBBBIB}.


\begin{savenotes}\sphinxattablestart
\centering
\begin{tabulary}{\linewidth}[t]{|T|T|T|T|T|T|T|}
\hline
\sphinxstyletheadfamily 
M
&\sphinxstyletheadfamily 
A
&\sphinxstyletheadfamily 
Ver
&\sphinxstyletheadfamily 
Med
&\sphinxstyletheadfamily 
CC
&\sphinxstyletheadfamily 
AES CTR
&\sphinxstyletheadfamily 
Pad
\\
\hline
2D2C
&
57686632
&
30
&
02
&
20
&
21870320
&
00
\\
\hline
\end{tabulary}
\par
\sphinxattableend\end{savenotes}


\subsection{Decrypted payload examples}
\label{\detokenize{omnipower:decrypted-payload-examples}}
The interpretation of the fields in the OmniPower are


\begin{savenotes}\sphinxattablestart
\centering
\begin{tabulary}{\linewidth}[t]{|T|T|T|T|T|T|}
\hline
\sphinxstyletheadfamily 
Field
&\sphinxstyletheadfamily 
Kamstrup name
&\sphinxstyletheadfamily 
Data fmt (DIF)
&\sphinxstyletheadfamily 
Value type (VIF/E)
&\sphinxstyletheadfamily 
VIF/E meaning
&\sphinxstyletheadfamily 
DIF VIF/E
\\
\hline
Data 1
&
A+
&
32\sphinxhyphen{}bit uint
&
Energy, 10\textasciicircum{}1 Wh
&
Consumption from grid, accum.
&
04  04
\\
\hline
Data 2
&
A\sphinxhyphen{}
&
32\sphinxhyphen{}bit uint
&
Energy, 10\textasciicircum{}1 Wh
&
Production to grid, accum.
&
04  84 3C
\\
\hline
Data 3
&
P+
&
32\sphinxhyphen{}bit uint
&
Power,  10\textasciicircum{}0 W
&
Consumption from grid, instantan.
&
04  2B
\\
\hline
Data 4
&
P\sphinxhyphen{}
&
32\sphinxhyphen{}bit uint
&
Power,  10\textasciicircum{}0 W
&
Production to grid, instantan.
&
04  AB 3C
\\
\hline
\end{tabulary}
\par
\sphinxattableend\end{savenotes}


\subsubsection{Decrypted short telegram}
\label{\detokenize{omnipower:decrypted-short-telegram}}

\begin{savenotes}\sphinxattablestart
\centering
\begin{tabulary}{\linewidth}[t]{|T|T|T|T|T|T|T|T|}
\hline
\sphinxstyletheadfamily 
CRC16
&\sphinxstyletheadfamily 
TPL\sphinxhyphen{}CI
&\sphinxstyletheadfamily 
Data fmt. sign.
&\sphinxstyletheadfamily 
CRC16 data
&\sphinxstyletheadfamily 
Data 1
&\sphinxstyletheadfamily 
Data 2
&\sphinxstyletheadfamily 
Data 3
&\sphinxstyletheadfamily 
Data 4
\\
\hline
1170
&
79
&
138C
&
4491
&
CE000000
&
00000000
&
03000000
&
00000000
\\
\hline
\end{tabulary}
\par
\sphinxattableend\end{savenotes}

Measurement data starts at byte 7, and can easily be extracted using \sphinxtitleref{\textless{}IIII} little\sphinxhyphen{}endian format.

In this example, 206 10\textasciicircum{}1 Wh (2.06 kWh) have been consumed, and the current power draw is 3 10\textasciicircum{}0 W (0.003 kW).


\subsubsection{Decrypted long telegram}
\label{\detokenize{omnipower:decrypted-long-telegram}}
In this kind of telegram, the DRHs are included.


\begin{savenotes}\sphinxattablestart
\centering
\begin{tabulary}{\linewidth}[t]{|T|T|T|T|T|T|T|T|T|T|}
\hline
\sphinxstyletheadfamily 
CRC16
&\sphinxstyletheadfamily 
TPL\sphinxhyphen{}CI
&\sphinxstyletheadfamily 
DIF/VIF 1
&\sphinxstyletheadfamily 
Data 1
&\sphinxstyletheadfamily 
DIF/VIF/VIFE 2
&\sphinxstyletheadfamily 
Data 2
&\sphinxstyletheadfamily 
DIF/VIF 3
&\sphinxstyletheadfamily 
Data 3
&\sphinxstyletheadfamily 
DIF/VIF 4
&\sphinxstyletheadfamily 
Data 4
\\
\hline
9831
&
78
&
04 04
&
D7000000
&
04 84 3C
&
00000000
&
04 2B
&
03000000
&
04 AB 3C
&
00000000
\\
\hline
\end{tabulary}
\par
\sphinxattableend\end{savenotes}

Extraction is slightly more complex, requiring either a longer parsing pattern or perhaps a regex.

In this example, 215 10\textasciicircum{}1 Wh (2.15 kWh) have been consumed, and the current power draw is 3 10\textasciicircum{}0 W (0.003 kW).


\section{The C1 Telegram class}
\label{\detokenize{omnipower:the-c1-telegram-class}}\index{C1Telegram (class in OmniPower.OmniPower)@\spxentry{C1Telegram}\spxextra{class in OmniPower.OmniPower}}

\begin{fulllineitems}
\phantomsection\label{\detokenize{omnipower:OmniPower.OmniPower.C1Telegram}}\pysiglinewithargsret{\sphinxbfcode{\sphinxupquote{class }}\sphinxcode{\sphinxupquote{OmniPower.OmniPower.}}\sphinxbfcode{\sphinxupquote{C1Telegram}}}{\emph{\DUrole{n}{telegram}\DUrole{p}{:} \DUrole{n}{\sphinxhref{https://docs.python.org/3.5/library/functions.html\#bytes}{bytes}}}}{}
Implements capture of data fields for a C1 telegram from OmniPower
\index{decrypt\_using() (OmniPower.OmniPower.C1Telegram method)@\spxentry{decrypt\_using()}\spxextra{OmniPower.OmniPower.C1Telegram method}}

\begin{fulllineitems}
\phantomsection\label{\detokenize{omnipower:OmniPower.OmniPower.C1Telegram.decrypt_using}}\pysiglinewithargsret{\sphinxbfcode{\sphinxupquote{decrypt\_using}}}{\emph{\DUrole{n}{meter}\DUrole{p}{:} \DUrole{n}{{\hyperref[\detokenize{omnipower:OmniPower.OmniPower.OmniPower}]{\sphinxcrossref{OmniPower.OmniPower.OmniPower}}}}}}{{ $\rightarrow$ \sphinxhref{https://docs.python.org/3.5/library/functions.html\#bool}{bool}}}
Decrypts a telegram using the key from the specified meter.
Updates the decrypted field of self.
Requires instantiated OmniPower meter with valid AES\sphinxhyphen{}key.

\end{fulllineitems}


\end{fulllineitems}



\section{The OmniPower class}
\label{\detokenize{omnipower:the-omnipower-class}}\index{OmniPower (class in OmniPower.OmniPower)@\spxentry{OmniPower}\spxextra{class in OmniPower.OmniPower}}

\begin{fulllineitems}
\phantomsection\label{\detokenize{omnipower:OmniPower.OmniPower.OmniPower}}\pysiglinewithargsret{\sphinxbfcode{\sphinxupquote{class }}\sphinxcode{\sphinxupquote{OmniPower.OmniPower.}}\sphinxbfcode{\sphinxupquote{OmniPower}}}{\emph{\DUrole{n}{name}\DUrole{p}{:} \DUrole{n}{\sphinxhref{https://docs.python.org/3.5/library/stdtypes.html\#str}{str}} \DUrole{o}{=} \DUrole{default_value}{\textquotesingle{}Kamstrup OmniPower one\sphinxhyphen{}phase\textquotesingle{}}}, \emph{\DUrole{n}{meter\_id}\DUrole{p}{:} \DUrole{n}{\sphinxhref{https://docs.python.org/3.5/library/stdtypes.html\#str}{str}} \DUrole{o}{=} \DUrole{default_value}{\textquotesingle{}32666857\textquotesingle{}}}, \emph{\DUrole{n}{manufacturer\_id}\DUrole{p}{:} \DUrole{n}{\sphinxhref{https://docs.python.org/3.5/library/stdtypes.html\#str}{str}} \DUrole{o}{=} \DUrole{default_value}{\textquotesingle{}2C2D\textquotesingle{}}}, \emph{\DUrole{n}{medium}\DUrole{p}{:} \DUrole{n}{\sphinxhref{https://docs.python.org/3.5/library/stdtypes.html\#str}{str}} \DUrole{o}{=} \DUrole{default_value}{\textquotesingle{}02\textquotesingle{}}}, \emph{\DUrole{n}{version}\DUrole{p}{:} \DUrole{n}{\sphinxhref{https://docs.python.org/3.5/library/stdtypes.html\#str}{str}} \DUrole{o}{=} \DUrole{default_value}{\textquotesingle{}30\textquotesingle{}}}, \emph{\DUrole{n}{aes\_key}\DUrole{p}{:} \DUrole{n}{\sphinxhref{https://docs.python.org/3.5/library/stdtypes.html\#str}{str}} \DUrole{o}{=} \DUrole{default_value}{\textquotesingle{}9A25139E3244CC2E391A8EF6B915B697\textquotesingle{}}}}{}
Implementation of our OmniPower single\sphinxhyphen{}phase meter
Passed values are hex encoded as string, e.g. ‘2C2D’ for value 0x2C2D.
\index{add\_measurement\_to\_log() (OmniPower.OmniPower.OmniPower method)@\spxentry{add\_measurement\_to\_log()}\spxextra{OmniPower.OmniPower.OmniPower method}}

\begin{fulllineitems}
\phantomsection\label{\detokenize{omnipower:OmniPower.OmniPower.OmniPower.add_measurement_to_log}}\pysiglinewithargsret{\sphinxbfcode{\sphinxupquote{add\_measurement\_to\_log}}}{\emph{\DUrole{n}{measurement}\DUrole{p}{:} \DUrole{n}{OmniPower.MeterMeasurement.MeterMeasurement}}}{{ $\rightarrow$ \sphinxhref{https://docs.python.org/3.5/library/constants.html\#None}{None}}}
Pushes a new measurement to the tail end of the log

\end{fulllineitems}

\index{decrypt() (OmniPower.OmniPower.OmniPower method)@\spxentry{decrypt()}\spxextra{OmniPower.OmniPower.OmniPower method}}

\begin{fulllineitems}
\phantomsection\label{\detokenize{omnipower:OmniPower.OmniPower.OmniPower.decrypt}}\pysiglinewithargsret{\sphinxbfcode{\sphinxupquote{decrypt}}}{\emph{\DUrole{n}{telegram}\DUrole{p}{:} \DUrole{n}{{\hyperref[\detokenize{omnipower:OmniPower.OmniPower.C1Telegram}]{\sphinxcrossref{OmniPower.OmniPower.C1Telegram}}}}}}{{ $\rightarrow$ \sphinxhref{https://docs.python.org/3.5/library/functions.html\#bytes}{bytes}}}~\begin{description}
\item[{Decrypt a telegram. Requires:}] \leavevmode\begin{itemize}
\item {} 
the prefix from the telegram (telegram.prefix), and

\item {} 
the encryption key from the meter.

\end{itemize}

\end{description}

Decrypts the data stored telegram.encrypted

\end{fulllineitems}

\index{dump\_log\_to\_json() (OmniPower.OmniPower.OmniPower method)@\spxentry{dump\_log\_to\_json()}\spxextra{OmniPower.OmniPower.OmniPower method}}

\begin{fulllineitems}
\phantomsection\label{\detokenize{omnipower:OmniPower.OmniPower.OmniPower.dump_log_to_json}}\pysiglinewithargsret{\sphinxbfcode{\sphinxupquote{dump\_log\_to\_json}}}{}{{ $\rightarrow$ \sphinxhref{https://docs.python.org/3.5/library/stdtypes.html\#str}{str}}}
Returns a JSON object of all measurement frames in log, with an incremented number for each observation

\end{fulllineitems}

\index{extract\_measurement\_frame() (OmniPower.OmniPower.OmniPower method)@\spxentry{extract\_measurement\_frame()}\spxextra{OmniPower.OmniPower.OmniPower method}}

\begin{fulllineitems}
\phantomsection\label{\detokenize{omnipower:OmniPower.OmniPower.OmniPower.extract_measurement_frame}}\pysiglinewithargsret{\sphinxbfcode{\sphinxupquote{extract\_measurement\_frame}}}{\emph{\DUrole{n}{telegram}\DUrole{p}{:} \DUrole{n}{{\hyperref[\detokenize{omnipower:OmniPower.OmniPower.C1Telegram}]{\sphinxcrossref{OmniPower.OmniPower.C1Telegram}}}}}}{{ $\rightarrow$ OmniPower.MeterMeasurement.MeterMeasurement}}
Requires that the telegram is already decrypted, otherwise returns empty measurement

\end{fulllineitems}

\index{is\_this\_my() (OmniPower.OmniPower.OmniPower method)@\spxentry{is\_this\_my()}\spxextra{OmniPower.OmniPower.OmniPower method}}

\begin{fulllineitems}
\phantomsection\label{\detokenize{omnipower:OmniPower.OmniPower.OmniPower.is_this_my}}\pysiglinewithargsret{\sphinxbfcode{\sphinxupquote{is\_this\_my}}}{\emph{\DUrole{n}{telegram}\DUrole{p}{:} \DUrole{n}{{\hyperref[\detokenize{omnipower:OmniPower.OmniPower.C1Telegram}]{\sphinxcrossref{OmniPower.OmniPower.C1Telegram}}}}}}{{ $\rightarrow$ \sphinxhref{https://docs.python.org/3.5/library/functions.html\#bool}{bool}}}
Check whether a given telegram is from this meter by comparing meter setting to telegram

\end{fulllineitems}

\index{process\_telegram() (OmniPower.OmniPower.OmniPower method)@\spxentry{process\_telegram()}\spxextra{OmniPower.OmniPower.OmniPower method}}

\begin{fulllineitems}
\phantomsection\label{\detokenize{omnipower:OmniPower.OmniPower.OmniPower.process_telegram}}\pysiglinewithargsret{\sphinxbfcode{\sphinxupquote{process\_telegram}}}{\emph{\DUrole{n}{telegram}\DUrole{p}{:} \DUrole{n}{{\hyperref[\detokenize{omnipower:OmniPower.OmniPower.C1Telegram}]{\sphinxcrossref{OmniPower.OmniPower.C1Telegram}}}}}}{{ $\rightarrow$ \sphinxhref{https://docs.python.org/3.5/library/functions.html\#bool}{bool}}}
Does entire processing chain for a telegram, including adding to log

\end{fulllineitems}

\index{unpack\_long\_telegram\_data() (OmniPower.OmniPower.OmniPower class method)@\spxentry{unpack\_long\_telegram\_data()}\spxextra{OmniPower.OmniPower.OmniPower class method}}

\begin{fulllineitems}
\phantomsection\label{\detokenize{omnipower:OmniPower.OmniPower.OmniPower.unpack_long_telegram_data}}\pysiglinewithargsret{\sphinxbfcode{\sphinxupquote{classmethod }}\sphinxbfcode{\sphinxupquote{unpack\_long\_telegram\_data}}}{\emph{\DUrole{n}{data}\DUrole{p}{:} \DUrole{n}{\sphinxhref{https://docs.python.org/3.5/library/functions.html\#bytes}{bytes}}}}{{ $\rightarrow$ Tuple\DUrole{p}{{[}}\sphinxhref{https://docs.python.org/3.5/library/functions.html\#int}{int}\DUrole{p}{, }\DUrole{p}{…}\DUrole{p}{{]}}}}
Long C1 telegrams contain DIF/VIF information and field data values

\end{fulllineitems}

\index{unpack\_short\_telegram\_data() (OmniPower.OmniPower.OmniPower class method)@\spxentry{unpack\_short\_telegram\_data()}\spxextra{OmniPower.OmniPower.OmniPower class method}}

\begin{fulllineitems}
\phantomsection\label{\detokenize{omnipower:OmniPower.OmniPower.OmniPower.unpack_short_telegram_data}}\pysiglinewithargsret{\sphinxbfcode{\sphinxupquote{classmethod }}\sphinxbfcode{\sphinxupquote{unpack\_short\_telegram\_data}}}{\emph{\DUrole{n}{data}\DUrole{p}{:} \DUrole{n}{\sphinxhref{https://docs.python.org/3.5/library/functions.html\#bytes}{bytes}}}}{{ $\rightarrow$ Tuple\DUrole{p}{{[}}\sphinxhref{https://docs.python.org/3.5/library/functions.html\#int}{int}\DUrole{p}{, }\DUrole{p}{…}\DUrole{p}{{]}}}}
Short C1 telegrams only contain field data values, no information about DIF/VIF

\end{fulllineitems}


\end{fulllineitems}



\chapter{Implementation of generic measurements}
\label{\detokenize{metermeasurement:module-OmniPower.MeterMeasurement}}\label{\detokenize{metermeasurement:implementation-of-generic-measurements}}\label{\detokenize{metermeasurement::doc}}\index{module@\spxentry{module}!OmniPower.MeterMeasurement@\spxentry{OmniPower.MeterMeasurement}}\index{OmniPower.MeterMeasurement@\spxentry{OmniPower.MeterMeasurement}!module@\spxentry{module}}

\section{Generic class for measurements and measurement frames}
\label{\detokenize{metermeasurement:generic-class-for-measurements-and-measurement-frames}}\begin{quote}\begin{description}
\item[{platform}] \leavevmode
Python 3.5.10 on Linux, OS X

\item[{synopsis}] \leavevmode
This module implements classes for generic measurements taken from a meter.

\item[{authors}] \leavevmode
Janus Bo Andersen, Jakob Aaboe Vestergaard

\item[{date}] \leavevmode
13 October 2020

\end{description}\end{quote}


\section{The Measurement class}
\label{\detokenize{metermeasurement:the-measurement-class}}\index{Measurement (class in OmniPower.OmniPower)@\spxentry{Measurement}\spxextra{class in OmniPower.OmniPower}}

\begin{fulllineitems}
\phantomsection\label{\detokenize{metermeasurement:OmniPower.OmniPower.Measurement}}\pysiglinewithargsret{\sphinxbfcode{\sphinxupquote{class }}\sphinxcode{\sphinxupquote{OmniPower.OmniPower.}}\sphinxbfcode{\sphinxupquote{Measurement}}}{\emph{\DUrole{n}{value}\DUrole{p}{:} \DUrole{n}{\sphinxhref{https://docs.python.org/3.5/library/functions.html\#float}{float}}}, \emph{\DUrole{n}{unit}\DUrole{p}{:} \DUrole{n}{\sphinxhref{https://docs.python.org/3.5/library/stdtypes.html\#str}{str}}}}{}
Single physical measurement.
A single measurement of a physical quantity pair, consisting of a value and a unit.

\end{fulllineitems}



\section{The MeterMeasurement class}
\label{\detokenize{metermeasurement:the-metermeasurement-class}}\index{MeterMeasurement (class in OmniPower.OmniPower)@\spxentry{MeterMeasurement}\spxextra{class in OmniPower.OmniPower}}

\begin{fulllineitems}
\phantomsection\label{\detokenize{metermeasurement:OmniPower.OmniPower.MeterMeasurement}}\pysiglinewithargsret{\sphinxbfcode{\sphinxupquote{class }}\sphinxcode{\sphinxupquote{OmniPower.OmniPower.}}\sphinxbfcode{\sphinxupquote{MeterMeasurement}}}{\emph{\DUrole{n}{meter\_id}\DUrole{p}{:} \DUrole{n}{\sphinxhref{https://docs.python.org/3.5/library/stdtypes.html\#str}{str}}}, \emph{\DUrole{n}{timestamp}\DUrole{p}{:} \DUrole{n}{\sphinxhref{https://docs.python.org/3.5/library/datetime.html\#datetime.datetime}{datetime.datetime}}}}{}
A single measurement collection based on one frame from the meter.
Will contain multiple measurements of physical quantities taken at the same time.
\index{add\_measurement() (OmniPower.OmniPower.MeterMeasurement method)@\spxentry{add\_measurement()}\spxextra{OmniPower.OmniPower.MeterMeasurement method}}

\begin{fulllineitems}
\phantomsection\label{\detokenize{metermeasurement:OmniPower.OmniPower.MeterMeasurement.add_measurement}}\pysiglinewithargsret{\sphinxbfcode{\sphinxupquote{add\_measurement}}}{\emph{\DUrole{n}{name}\DUrole{p}{:} \DUrole{n}{\sphinxhref{https://docs.python.org/3.5/library/stdtypes.html\#str}{str}}}, \emph{\DUrole{n}{measurement}\DUrole{p}{:} \DUrole{n}{OmniPower.MeterMeasurement.Measurement}}}{{ $\rightarrow$ \sphinxhref{https://docs.python.org/3.5/library/constants.html\#None}{None}}}
Store a new measurement in the collection.

\end{fulllineitems}

\index{as\_dict() (OmniPower.OmniPower.MeterMeasurement method)@\spxentry{as\_dict()}\spxextra{OmniPower.OmniPower.MeterMeasurement method}}

\begin{fulllineitems}
\phantomsection\label{\detokenize{metermeasurement:OmniPower.OmniPower.MeterMeasurement.as_dict}}\pysiglinewithargsret{\sphinxbfcode{\sphinxupquote{as\_dict}}}{}{{ $\rightarrow$ \sphinxhref{https://docs.python.org/3.5/library/stdtypes.html\#dict}{dict}}}
Serializes and dumps the Measurement frame as a dict.
Make an object similar to
\{
\begin{quote}

“Meter ID: “: “3232323”,
“Timestamp:”: “2020\sphinxhyphen{}10\sphinxhyphen{}13T17:36:53”,
“Measurements”: \{
\begin{quote}
\begin{description}
\item[{“A+”: \{}] \leavevmode
“unit”: “kWh”,
“value”: 7

\end{description}

\},
“A\sphinxhyphen{}“: \{
\begin{quote}

“unit”: “kWh”,
“value”: 8
\end{quote}

\},
“P+”: \{
\begin{quote}

“unit”: “kW”,
“value”: 9
\end{quote}

\},
“P\sphinxhyphen{}“: \{
\begin{quote}

“unit”: “kW”,
“value”: 10
\end{quote}

\}
\end{quote}

\}
\end{quote}

\}

\end{fulllineitems}

\index{json\_dump() (OmniPower.OmniPower.MeterMeasurement method)@\spxentry{json\_dump()}\spxextra{OmniPower.OmniPower.MeterMeasurement method}}

\begin{fulllineitems}
\phantomsection\label{\detokenize{metermeasurement:OmniPower.OmniPower.MeterMeasurement.json_dump}}\pysiglinewithargsret{\sphinxbfcode{\sphinxupquote{json\_dump}}}{}{{ $\rightarrow$ \sphinxhref{https://docs.python.org/3.5/library/stdtypes.html\#str}{str}}}
Returns a JSON formatted string of all data in frame.

\end{fulllineitems}


\end{fulllineitems}



\chapter{Indices and tables}
\label{\detokenize{index:indices-and-tables}}\begin{itemize}
\item {} 
\DUrole{xref,std,std-ref}{genindex}

\item {} 
\DUrole{xref,std,std-ref}{modindex}

\item {} 
\DUrole{xref,std,std-ref}{search}

\end{itemize}


\renewcommand{\indexname}{Python Module Index}
\begin{sphinxtheindex}
\let\bigletter\sphinxstyleindexlettergroup
\bigletter{o}
\item\relax\sphinxstyleindexentry{OmniPower.MeterMeasurement}\sphinxstyleindexpageref{metermeasurement:\detokenize{module-OmniPower.MeterMeasurement}}
\item\relax\sphinxstyleindexentry{OmniPower.OmniPower}\sphinxstyleindexpageref{omnipower:\detokenize{module-OmniPower.OmniPower}}
\end{sphinxtheindex}

\renewcommand{\indexname}{Index}
\printindex
\end{document}